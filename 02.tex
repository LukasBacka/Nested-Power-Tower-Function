
\section{}

\subsection{Series Representation}

\subsubsection{nth derivation of the \(\mathcal{T}\)-function }

to complete the final formula for the limit (\ref{nn}),
we are forced to find a general formula for the nth 
derivative of the \(\mathcal{T}\)-function

\begin{align}
        \frac{d^\xi}{dx^\xi}(\mathcal{T}(a,| x) ) & =a^x \ln^\xi(a)\\
        \frac{d^\xi}{dx^\xi}(\mathcal{T}(a, b | x) ) &= T(a, b |x) 
        \ln^\xi(b) \sum_{w=1}^\xi \ln^w(a) b^{wx} S_2(w, \xi) \\
        \frac{d^\xi}{dx^\xi}(\mathcal{T}(a, b, c | x) ) &= 
        \mathcal{T}(a, b, c | x) \ln^\xi(c) \left[ \sum_{i=1}^\xi 
        S_2(i, \xi) \ln^i(b)c^{ix} \sum_{j=1}^i \ln^j(a) b^{jc^x}\right]
\end{align}

where \(S_2(w, \xi)\) are second order Stirling numbers. 
Then through empirical observation, we establish the following 
hypothesis

\begin{theorem}

        \begin{align}
                \frac{d^\xi}{dx^\xi}(\mathcal{T}(\bigotimes_{m \geq t 
                \geq n}x_t | x) ) &= \mathcal{T}(\bigotimes_{m \geq t 
                \geq n}x_t| x) \ln^\xi (x_m) \left[\sum_{u_1 = 1}^\xi 
                S_2(u_1, \xi) \ln^{u_1}(x_{m-1}) \mathcal{T}(x_{m}| u_1  x) 
                \times \right. \\ & \left. \times \sum_{u_2 = 1}^{u_1} 
                \ln^{u_2}(x_{m-2}) \mathcal{T}(x_{m-1} u_2, x_m |x) 
                \sum_{u_3}^{u_2}...\right. \\ & ... \left. \sum_{u_{m-1}
                =1}^{u_{m-2}}\ln^{u_{m-1}}(x_1)\mathcal{T}(\bigotimes_{q=2}^m
                (x_q + \delta_{m, q}(-x_q + x_q u_{m-1})) |x)\right]
        \end{align}

        by means of noncommutative multiplication, which we denote by 
        \(o\), we can reformulate the above relation with \(u_0 = \xi\)
        
        \begin{align}
                \frac{d^\xi}{dx^\xi}(\mathcal{T}(\bigotimes_{m \geq t \geq n}x_t | x) ) 
                &= \mathcal{T}(\bigotimes_{m \geq t \geq n}x_t| x) \ln^\xi (x_m) 
                \biigO_{\lambda = 1}^{m-1} R\left[\sum_{u_{\lambda}=1}^{u_{\lambda-1}}
                \ln^{u_{\lambda}}(x_{m-\lambda}) \mathcal{T}(\bigotimes_{q = \lambda-1}^\lambda
                x_{m-q} |x)^{u_{\lambda}} \right]
        \label{w}
        \end{align}
        
        \begin{align}
                R=\left[ \lceil \tanh^2(\lambda -1)\rceil+S_{\lambda, 1}(S_2((u_1, \xi))\right]
        \end{align}
        
\end{theorem}

\begin{align}
 \mathcal{T}\left(\bigotimes_{n=1}^ka_n | x\right) = \sum_{n=0}^\infty 
 \frac{1}{n!}\left(\biigO_{l=1}^{k-1}\ln(a_l)\left[\sum_{n_l\geq0}\frac{1}
 {n_l!}\left(\lambda_l\right)^{n_l}\right]^{\Lambda(l, k, \ln(a_k)x}\right)^n
\end{align}

where \(\Lambda(l, k, \ln(a_k)x) :=\left[\lceil tanh^2(l-k+1)\rceil+
\delta_{l, k-1}(\ln(a_k)x)\right]\), this function is equal to \(\ln(a_k)x\)
iff \(l=k-1\), in other cases, its 0. Then higher derivation of this function is

\begin{align}
        \frac{d^\xi\mathcal{T}}{dx^\xi}\left(\bigotimes_{n=1}^ka_n | x\right) 
        = \sum_{n\geq0}\frac{1}{n!}\ln^n(a_1) \frac{d^\xi}{dx^\xi}\left(\sum_{n_1\geq0}
        \frac{1}{n_1!}g(x)\right)
\end{align}

We use the Multinomial formula for

\begin{align}
    \frac{d^\xi}{dx^\xi}\left(\sum_{n_1\geq0}\frac{1}{n_1!}g(x)\right)
    &=  \frac{d^\xi}{dx^\xi} \sum_{n_1\geq0} \left(\sum_{\substack{0\leq
    r_1 \leq...\leqr_n\leq n_1 \\ r_1 +...+r_n = n_1}}\prod_{i=1}^n 
    \frac{1}{r_i}\right)g^{n_1}(x) \\
    &=\sum_{n_1\geq0} \left(\sum_{\substack{0\leq r_1 \leq...\leqr_n\leq
    n_1 \\ r_1 +...+r_n = n_1}}\prod_{i=1}^n \frac{1}{r_i}\right)
    \frac{d^\xi}{dx^\xi}g^{n_1}(x)
\end{align}

There is set \(S=\{r_1, ..., r_n\}\) of many distinct elements with 
\(m \leq n\), call these distinct elements \(s_1, ..., s_m\). And define
\(N(s_i) = \# \mbox{of} \ s_i \ \mbox{appears in} \ (r_1, ..., r_n) \), 
then we see that the number of time \(a_{r_1}...a_{r_n}\) is is counted is equal to 

\begin{align}
    \frac{n!}{N(s_1)!...N(s_m)!}
\end{align}

thus we have

\begin{align}
    \sum_{n_1\geq0} \left(\sum_{\substack{0\leq r_1 \leq...\leqr_n\leq
    n_1 \\ r_1 +...+r_n = n_1}}\frac{n!}{N(s_1)!...N(s_m)!}\prod_{i=1}^n 
    \frac{1}{r_i}\right)\frac{d^\xi}{dx^\xi}g^{n_1}(x)
\end{align}

hence

\begin{align}
      \frac{d^\xi}{dx^\xi}\left(\sum_{n_1\geq0}\frac{1}{n_1!}g(x)\right)
      &=\sum_{n\geq0}\frac{1}{n!}\ln^n(a_1) \frac{d^\xi}{dx^\xi}\left(
      \sum_{n_1\geq0}\frac{1}{n_1!}g(x)\right) \\
      &=\sum_{n\geq 0}\frac{1}{n!}\ln^n(a_1)\sum_{n_1\geq0} 
      \left(\sum_{\substack{0\leq r_1 \leq...\leqr_n\leq n_1 
      \\ r_1 +...+r_n = n_1}}\prod_{i=1}^n \frac{1}{r_i}\right)
      \frac{d^\xi}{dx^\xi}g^{n_1}(x) \\
      & =\sum_{n_0\geq 0}\frac{1}{n_0!}\biigO_{l=1}^{k-1}
      \ln^{n_{l-1}}(a_l)
      \sum_{n_l\geq0} \left(\sum_{\substack{0\leq r_1 \leq...\leqr_n\leq
      n_l \\ r_1 +...+r_n = n_l}}\frac{n_{l-1}!}{N(s_1)!...N(s_m)!}
      \prod_{i=1}^{n_{l-1}} \frac{1}{r_i}\right)(\lambda_l)^{n_l 
      \Lambda(l, k,\frac{\ln(a_k)x^{-\xi+n_{l-1}}}{((1-\xi + n_{l-1})_\xi)^{-1}}}
\end{align}

then we use equation (\ref{w}), so we get

\begin{align}
        \mathcal{T}(\bigotimes_{k \geq t \geq 1}a_t| x) \ln^\xi (a_k) 
        \biigO_{\lambda = 1}^{m-1} R\left[\sum_{u_{\lambda}=1}^{u_{\lambda-1}}
        \ln^{u_{\lambda}}(a_{k-\lambda}) \mathcal{T}(\bigotimes_{q = \lambda-1}^\lambda
        a_{k-q} |x)^{u_{\lambda}} \right] = \\=\sum_{n_0\geq 0}\frac{1}{n_0!}\biigO_{l=1}^{k-1}
        \ln^{n_{l-1}}(a_l) \sum_{n_l\geq0} \left(\sum_{\substack{0\leq r_1 \leq...\leqr_n\leq 
        n_l \\ r_1 +...+r_n = n_l}}\frac{n_{l-1}!}{N(s_1)!...N(s_m)!}\prod_{i=1}^{n_{l-1}} 
        \frac{1}{r_i}\right)(\lambda_l)^{n_l \Lambda(l, k,\frac{\ln(a_k)
        x^{-\xi+n_{l-1}}}{((1-\xi + n_{l-1})_\xi)^{-1}}}
\end{align}

we solve this equation for \( \mathcal{T}(\bigotimes_{k \geq t \geq 1}a_t| x)\)

\begin{align}
        \mathcal{T}(\bigotimes_{k \geq t \geq 1}a_t| x) &=\left(\ln^\xi (a_k) 
        \biigO_{\lambda = 1}^{k-1} \left[ \lceil \tanh^2(\lambda -1)\rceil+
        \delta_{\lambda, 1}(S_2((u_1, \xi))\right]\left[\sum_{u_{\lambda}=1}^{u_{\lambda-1}}
        \ln^{u_{\lambda}}(a_{k-\lambda}) \mathcal{T}(\bigotimes_{q = \lambda-1}^\lambda 
        a_{k-q} |x)^{u_{\lambda}}\lambda_\lambda \right]\right)^{-1}\times \\ &\times 
        \sum_{n_0\geq 0}\frac{1}{n_0!}\biigO_{l=1}^{k-1}
        \ln^{n_{l-1}}(a_l) \sum_{n_l\geq0} \left(\sum_{\substack{0\leq
        r_1 \leq...\leq r_n\leq n_l \\ r_1 +...+r_n = n_l}}\frac{n_{l-1}!}{N(s_1)
        !...N(s_m)!}\prod_{i=1}^{n_{l-1}} \frac{1}{r_i}\right)(\lambda_l)^{n_l 
        \Lambda \left(l, k,\frac{\ln(a_k)x^{-\xi+n_{l-1}}}{((1-\xi + n_{l-1})_\xi)^{-1}}\right)}
\end{align}

\subsubsection{Exponential Factorial}

The exponential factorial is defined by the recurrence relation

\begin{align}
        a_n=n^{a_{n-1}}, 	
\end{align}

where \(a_0=1\). The first few terms are therefore

\begin{align}
        a_1	&=	1	 \\
        a_2	&=	2^1=2	\\
        a_3	&=	3^{2^1}=3^2=9	\\
        a_4	&=	4^{3^{2^1}}=4^9=262144	\\
\end{align}

(OEIS A049384). The term \(a_5=5^(262144)\) has 183231 digits. The exponential 
factorial is therefore a kind of "factorial power tower." The sum of the reciprocals 
of the exponential factorials is given by

\begin{align}
        S	&=	\sum_{k=1}^{\infty}\frac{1}{a_k}	\\
        &=	1.61111492580837673611...
\end{align}

(OEIS A080219). This sum is a Liouville number and is therefore transcendental.
Like tetration, there is currently no accepted method of extension of the exponential
factorial function to real and complex values of its argument, unlike the factorial 
function, for which such an extension is provided by the gamma function. But it is 
possible to expand it if it is defined in a strip width of 1. Then we can express 
Exponential factorial by the Tower function as.

\begin{align}
        a_s &= \mathcal{T}\left(\bigotimes_{i=1}^s(s-i+1)|1\right) \\
        & = \left(\ln^\xi (s-k+1) \biigO_{\lambda = 1}^{s-1} \left[ \lceil 
        \tanh^2(\lambda -1)\rceil+\delta_{\lambda, 1}(S_2((u_1, \xi))\right]
        \left[\sum_{u_{\lambda}=1}^{u_{\lambda-1}}\ln^{u_{\lambda}}(\lambda+1) 
        \mathcal{T}(\bigotimes_{q = \lambda-1}^\lambda (q+1) |x)^{u_{\lambda}}
        \lambda_\lambda \right]\right)^{-1}\times \\ &\times \sum_{n_0\geq 0}
        \frac{1}{n_0!}\biigO_{l=1}^{s-1} \ln^{n_{l-1}}(s-l+1)
        \sum_{n_l\geq0} \left(\sum_{\substack{0\leq r_1 \leq...\leq r_n\leq n_l 
        \\ r_1 +...+r_n = n_l}}\frac{n_{l-1}!}{N(s_1)!...N(s_m)!}\prod_{i=1}^{n_{l-1}}
        \frac{1}{r_i}\right)(\lambda_l)^{n_l \Lambda \left(l, s,0\right)}
\end{align}
